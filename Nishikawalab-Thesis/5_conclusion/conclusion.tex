本研究では,ヒトと同じ自由度を持った足関節ロボット,伸縮性に富んだフレキシブルストレッチセンサ
(ストレッチセンサ)の開発製作を行った.

足関節ロボットにおいて従来では底背屈動作しか行うことが出来なかったものが,ヒトの足関節と同様の
自由度を持った3自由度の動きをすることが,可能になった.また,この自由度で動作することを
,腓骨筋とヒラメ筋,前脛骨筋の組み合わせで動作させ確認することができた.

現在,足関節ロボットは左足のみであるが既存の2足歩行ロボットの様に両足を備えた骨盤より下の
下肢ロボットになるよう製作を行っていく必要があると思われる.これによって実際のヒトと同じ
自由度を持った空気圧人工筋を用いた2足歩行ロボットとなり,ヒトの運動戦略の解析に役立つと
考えられる為である.

ストレッチセンサにおいて,実際に計測回路も含めて製作し空気圧人工筋に搭載した.そして,
空気圧人工筋の伸縮に関して計測を行うことが出来た.また,周期動作を行っている空気圧人工筋の
伸縮に関して計測を行うこともできた.既製品では5万円程かかる所を,
計測回路も含め6千円程度で製作した.これにより,安価に多チャンネルの計測を行うことができる
様になった.

今後の課題として,ストレッチセンサのさらなる実用化を行っていく必要があると思われる.具体的には,
ロボットの制御系にストレッチセンサの計測系を組み込み,フィードバック系として活用できる様に
することが一つとして挙げられる.これに備えて,現在では計測後にデータ処理等を行っていたが,
リアルタイムにデータ処理を行えるように組む必要性があると考えられる.また,現在は製作したセンサごとの
特性のばらつきが大きいが定量的に製作できるようにして,センサごとの特性を小さくしていく必要がある.
さらに,計測回路もより高精度にデータ取得できるよう改良を施す余地が見られる.
